\documentclass{dstu}
\usepackage{minted}
\setminted{
	breaklines,
	linenos,
	frame=lines,
	fontsize=\small,
	baselinestretch=1,
	fontfamily=tt
}
% did not work...
% \newcommand{\withfname}[2]{file: #1\par\inputminted{#1}}

% Custom lists
\setlist[enumerate,1]{label=\arabic*), partopsep=0pt, parsep=0pt, topsep=0pt, itemsep=0pt, itemindent=1.9cm, leftmargin=0pt}
\setlist[enumerate,2]{label=\roman*), partopsep=0pt, parsep=0pt, topsep=0pt, itemsep=0pt, itemindent=2.6cm, leftmargin=0pt}

\begin{document}

\maketitlepage{
	StudentName={Подолянко Т.О.},
	StudentMale=true,
	StudentGroup={ФБ-11},
	Title={Розрахункова робота\\із дисципліни \invcommas{Захист програмного забезпечення} на тему\\Розробка програми автентифікації користувача по клавіатурному почерку},
	SupervisorDegree={доцент},
	SupervisorName={Коломицев М. В.}
}

\tableofcontents

\chapter{Постановка завдання}

Розробити програму автентифікації користувача по клавіатурному почерку.
Вимоги до програми:
\begin{enumerate}{}
	\item{
		Програма повинна працювати в двох режимах:
		\begin{itemize}
			\item навчання (створення біометричного еталону)
			\item  ідентифікації (порівняння з біометричним еталоном).
		\end{itemize}
	}

	\item На етапі навчання необхідно визначати еталонні статистичні параметри 
	клавіатурного почерку – оцінки математичного чекання і дисперсії 
	тривалості утримання клавіш. Параметри повинні записуватися у файл. 
	Вчення виробляти по багатократному набору фіксованій контрольній фразі, 
	символи якої рівномірно розподілені по клавіатурі.

	\item На етапі ідентифікації необхідно визначити параметри введеної 
контрольної фрази і перевірити гіпотезу про те, що отримані оцінки 
математичного очікування і дисперсії належать тому ж розподілу, що і 
параметри біометричного еталону. На цьому етапі необхідно відображувати 
отримані оцінки і еталонні параметри. Рівень значущості критерію задавати в 
діалоговому вікні.

	\item На етапах навчання і ідентифікації передбачити можливість 
відбракування грубих помилок (окремих вимірів)

\end{enumerate}

\chapter{Алгоритм вирішення завдання}

Відомо, що часові інтервали між натисканням сусідніх букв при наборі ключової фрази,
як правило, підпорядковуються нормальному закону розподілення. Для нормального
розподілу випадкової величини існують алгоритми для:

\begin{itemize}
	\item побудови довірчих інтервалів математичного очікування і дисперсії;

	\item перевірки гіпотези про рівність центрів розподілу двох нормальних 
	генеральних сукупностей при допущенні про рівність дисперсій;

	\item перевірки гіпотези про рівність дисперсій двох нормальних 
	генеральних сукупностей;
	
	\item виключення грубих помилок у спостереженні
\end{itemize}

\bigskip
Отже, алгоритм роботи програми в режимі навчання такий:

\begin{enumerate}
	\item Запросити у користувача параметри для навчання — ключову фразу, кількість $k_e$ еталонних вимірів, які будуть збережені
	\item Переконатися, що такий користувач ще не існує
	\item Зчитати ключову фразу $k_e$ разів, реєструючи часові інтервали між натисканнями
	\item Для кожної послідовності часових інтервалів виключити грубі помилки за алгоритмом виключення грубих помилок
	\item Для кожної послідовності часових інтервалів $y_i = \{y_1, y_2, ...,, y_n\}, \text{$n$ - довжина ключової фрази}$ розрахувати 
	математичне сподівання $M_i$ та дисперсію $S^2_i$, $i = 1..n$
	\item Зберегти модель у вигляді множини пар
	($M_i$, $S^2_i$) та ключової фрази у базі користувачів.
\end{enumerate}

Нехай $p_a$ - мінімальний поріг ймовірності того, що поточний користувач є автором еталонних характеристик, за якого 
автентифікація вважається успішною; $\alpha_1$ - рівень значимості для задачі перевірки однорідності двох вибіркових дисперсій;
$\alpha_2$ - рівень значимості для задачі перевірки гіпотези про рівність центрів розподілу двох 
нормальних генеральних сукупностей. Тоді алгоритм автентифікації користувача наступний:

\begin{enumerate}
	\item Запросити у користувача параметри автентифікації — ім'я користувача
	\item Зчитати ключову фразу, реєструючи часові інтервали між натисканнями
	\item Відкинути незначущі значення за алгоритмом виключення грубих помилок
	\item Для кожної еталонної характеристики користувача, на ім'я якого виконується автентифікація:
	\begin{enumerate}
		\item Перевірити гіпотезу про однорідність вибіркових дисперсій
		— еталонної та зчитаної послідовностей інтервалів. Якщо гіпотеза не виконується для рівня значущості $\alpha_1$, то вважаємо,
		що зразок не відповідає еталонній характеристиці.
		\item Перевірити гіпотезу про рівність центрів розподілу двох нормальних генеральних сукупностей
		— еталонної та зчитаної послідовностей інтервалів. Якщо гіпотеза не виконується для рівня значущості $\alpha_2$, то вважаємо,
		що зразок не відповідає еталонній характеристиці.
		\item Якщо обидві гіпотези приймаються для відповідних рівнів значущості, то вважаємо,
		що зразок відповідає еталонній характеристиці.
	\end{enumerate}
	\item Розраховуємо $p = r / k_e$, де $r$ - кількість еталонних характеристик, які відповідають зразку.
	\item Якщо $p >= p_a$, то автентифікація вважається успішною.
\end{enumerate}

\begin{figure}[!htp]
	\centering
	\includegraphics[scale=0.5]{img/scheme.png}
	\caption{Блок-схема алгоритму роботи програми}
	\label{fig:figure1}
\end{figure}

\bigskip
Для перевірки статистичних гіпотез використовується наступний математичний апарат:

\begin{equation}
	\label{eq:1}
	\text{Вибіркове математичне сподівання: }
	M_i = \frac{\sum^{n}_{k=1}{y_{i_{k}}}}{n}
\end{equation}

\begin{equation}
	\label{eq:2}
	\text{Вибіркоа дисперсія: }
	S_i^2 = \frac{\sum^{n}_{k=1}{(y_{i_{k}} - M_i)^2}}{n-1}
\end{equation}

\begin{equation}
	\label{eq:3}
	\text{Середньоквадратичне відхилення: }
	S_i = \sqrt(S_i^2)
\end{equation}

\begin{equation}
	\label{eq:4}
	\text{Коефіцієнт Стьюдента: }
	t_p = |{\frac{y_i - M_i}{S_i}}|
\end{equation}

Нехай $y = \lbrace{y_1, y_2, ..., y_n}\rbrace$ - вибірка.
Алгоритм виключення грубих помилок у спостереженні наступний:

\begin{enumerate}
	\item Позначимо $y'=y \backslash y_i$ — $y$ без елемента $y_i$
	\item Розрахувати вибіркове матсподівання $M$ для $y'$ за \ref{eq:1}
	\item Розрахувати вибіркову дисперсію $S^2$ для $y'$ за \ref{eq:2}
	\item Розрахувати середньоквадратичне відхилення для $y'$ за \ref{eq:3}
	\item Розраховуємо коефіцієнт Стьюдента $t_p$ за \ref{eq:4}
	\item Встановимо табличне значення коефіцієнта Стьюдента $t_t$ для $n-1$ ступенів свободи з обраним рівнем значущості $\alpha_1$.
	Якщо $t_p > t_t$, то $y_i$ є виміром з грубою похибкою.
\end{enumerate}

Нехай $y = \lbrace{y_1, y_2, ..., y_n}\rbrace, x = \lbrace{x_1, x_2, ..., x_n}\rbrace$ - дві незалежні вибірки,
$S^2_1, S^2_2$ — їх відповідні дисперсії.
Алгоритм перевірки гіпотези про однорідність дисперсій наступний:

\begin{enumerate}
	\item Позначимо $S^2_{max} = max(S^2_1, S^2_2), S^2_{min} = min(S^2_1, S^2_2)$
	\item Обчислимо коефіцієнт Фішера $F_p = \frac{S^2_{max}}{S^2_{min}}$
	\item Встановимо табличне значення Фішера $F_t$ для $n-1$ ступенів свободи з обраним рівнем значущості $\alpha_2$.
	Якщо $F_p > F_t$, то гіпотеза про однорідність відкидається.
\end{enumerate}

Нехай $M_1, M_2$ — вибіркові математичні сподівання відповідно двох незалежних вибірок,
$S^2_1, S^2_2$ вибіркові дисперсії відповідно тих самих вибірок.
Алгоритм перевірки гіпотези про рівність центрів розподілів двох 
нормальних генеральних сукупностей:

\begin{enumerate}
	\item Обчислимо $S = \sqrt{ \frac{(S^2_1)^2 + (S^2_2)^2 \cdot (n-1)}{2n-1} }$ 
	\item Знайдемо коефіцієнт Стьюдента за формулою $t_p = \frac{|M_1 - M_2|}{S \cdot \sqrt{\frac{2}{n}}}$
	\item Встановимо табличне значення коефіцієнта Стьюдента $t_t$ для $n-1$ ступенів свободи з обраним рівнем значущості $\alpha_3$.
	Якщо $t_p > t_t$, то гіпотеза про рівність центрів розподілу двох 
	нормальних генеральних сукупностей відкидається.
\end{enumerate}

Оцінка ймовірності того, що автор зразка є автором еталонних характеристик —
$p = r / k_e$, де $r$ - кількість еталонних характеристик, для яких приймається
гіпотези про однорідність дисперсій та про рівність центрів розподілів зі зразком.

\chapter{Результати експериментальних досліджень}

Для оцінки помилок першого та другого роду до програми додано функцію збирання статистики.
Для оцінки похибок два користувачі намагалися автентифікуватися до одного акаунту.
Нелегітимний користувач здійснив 15 спроб входу, всі вони були неуспішними.
Легітимний користувач з 45 спроб входу здійснив 29 успішно.

\begin{figure}[!htp]
	\centering
	\includegraphics[scale=0.5]{img/stats.png}
	\caption{Зібрана статистика (друга колонка — легітимний користувач, чи ні; третя — к-ть успіхів; четверта — к-ть відмов)}
	\label{fig:figure1}
\end{figure}

\begin{figure}[!htp]
	\centering
	\includegraphics[scale=0.5]{img/farfrr.png}
	\caption{Вимір FRR/FAR}
	\label{fig:figure2}
\end{figure}

Після тестування двома користувача результат наступний:

\begin{itemize}
	\item Помилки першого роду (FRR) — 35.56\%
	\item Помилки другого роду (FAR) — 0.0\%
\end{itemize}

\chapter{Приклад виконання і лістинг функціональної частини програми}

\begin{figure}[!htp]
	\centering
	\includegraphics[scale=0.5]{img/run1.png}
	\caption{Демонстрація роботи програми 1}
	% \label{fig:figure3}
\end{figure}

\begin{figure}[!htp]
	\centering
	\includegraphics[scale=0.5]{img/run2.png}
	\caption{Демонстрація роботи програми 2}
	% \label{fig:figure3}
\end{figure}

Реалізована програма спирається на функціонал математичного пакету scipy,
в якому реалізовані функції для роботи з t-розподілом Стьюдента та розподілом Фішера.
Також використовується pydantic.

Вихідний код також розміщено на Github за посиланням
\href{https://github.com/timofey282228/appsec-project}{https://github.com/timofey282228/appsec-project}

\inputminted{python}{../appsec_project/win_read_cli.py}
\inputminted{python}{../appsec_project/common_read.py}
\inputminted{python}{../appsec_project/model.py}
\inputminted{python}{../appsec_project/statutils.py}
\inputminted{python}{../appsec_project/storage.py}
\inputminted{python}{../appsec_project/__main__.py}

\end{document}
